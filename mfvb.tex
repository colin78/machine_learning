%!TEX root = paper.tex

\section{Mean-Field Variational Bayes}\label{sec:mfvb}

Mean-field variational Bayes (MFVB) is a method for approximating the posterior distribution.  In general, we have unknown parameters $w_1, w_2, \ldots, w_n$ that we have priors on, and our objective is to find the joint distribution $p(w_1, w_2, \ldots, w_n)$.  Assuming that our approximate distribution is in the family $Q = \{q : q(w_1, w_2, \ldots, w_n) = q(w_1)q(w_2) \ldots q(w_n)\}$, we find $q^* \in Q$ that minimizes the KL-divergence with $p$, i.e. $q^* = \min KL(q || p)$. 
In particular, for logistic regression, the analytical form of the posterior is unknown and has been approximated with MFVB in the literature. We use local variational bounds on the conditional probability using the convexity of the logarithm function. In particular, we use a variational treatment based on the approach of Jaakkola and Jordan (2000). 
This approach consists of approximating the likelihood function of the logistic regression, governed by the sigmoid function, by the exponential of the quadratic form, leading to a gaussian approximation of the posterior distribution. More explicitly, if $y\in \{-1,1\}$ is a target variable for a data vector $x$ then the likelihood function of the target variable $y$ is: 
\begin{equation}
p(y | x, w)=\sigma(y w^T x)
\end{equation}

with $w$ being the logistic regression weight, and $\sigma(x)=\dfrac{1}{1+\exp(-x)}$ the sigmoid function. 
Using a transformation of the logarithm of the sigmoid and the concept of convex duality, we get: 
\begin{equation}
\sigma(x) \geq \sigma(\xi)\exp((x-\xi)/2-\lambda(\xi)(x^2-\xi^2))
\end{equation}

where
$$\lambda(\xi)=\frac{1}{2\xi}[\sigma(\xi)-\frac{1}{2}]=\frac{1}{4\xi}\tanh(\frac{\xi}{2})$$
and $\xi$ is a variational parameter. 

Therefore, if we let $a=w^T x$ we get: 
\begin{equation}
p( y | x,w)\geq e^{ya} \sigma(\xi)\exp\{-(\xi+a)/2-\lambda(\xi)(a^2-\xi^2)\}
\end{equation}

To every training set observation $(x_n, y_n)$, there is a variational parameter $\xi_n$ associated. We apply the bound above to each of the terms in the likelihood function. Let $Y=[y_1, y_2, \ldots , y_n]^T$ and $X$ be the data matrix. Then the likelihood function is: 
\begin{equation}
p( Y | X, w)=\prod_{i=1}^{N} p(y_i | x_i, w) = \prod_{i=1}^{N} \sigma(y w^T x)
\end{equation}

and thus we obtain the following bound on the marginal data likelihood:
\begin{equation}
p( Y | X, w) \geq h(w, \xi)
\end{equation}
and 
$$h(w,\xi) = \prod_{i=1}^{N} e^{y_iw^T x_i} \sigma(\xi_i)\exp\{-(\xi_i+w^T x_i)/2-\lambda(\xi_i)((w^T x_i)^2-{\xi_i}^2)\}$$

This approximation is used because the sigmoid data likelihood does not have a conjugate in the exponential family of priors. The approximation we obtain is quadratic in $w$ in the exponential, and we use the conjugate gaussian prior:
\begin{equation}
p( w | \alpha) = \mathcal{N}(0, \alpha ^{-1} \mathcal{I})
\end{equation}
and we can also model the hyper-parameter $\alpha$ with a conjugate Gamma distribution: 
\begin{equation}
p(\alpha) = Gam( \alpha | a_0, b_0)
\end{equation}

Variational Bayesian inference aims at maximizing a lower bound of the data log-likelihood. The log-likelihood is
\begin{equation}
\ln p(Y| X)=\ln \int\int p(Y| X, w) p(w |\alpha)p(\alpha)dw d\alpha
\end{equation}

We approximate the posterior $p(w, \alpha | X)$ by the variational distribution $Q(w, \alpha)$ that can be factorized to $Q(\alpha)Q(w)$. The bound of the log-likelihood is as follows: 
\begin{equation}
\ln p(Y| X) \geq \mathcal{L}(Q)=\ln \int\int Q(w, \alpha) \ln \dfrac{p(Y| X, w) p(w |\alpha)p(\alpha)}{Q(w,\alpha)}dw d\alpha
\end{equation}

Hence, using Equation (5), we obtain a variational bound $ \mathcal{\tilde{L}}(Q, \xi) $ that we aim at maximizing:
\begin{equation}
\mathcal{\tilde{L}}(Q, \xi)=\int\int Q(w, \alpha) \ln \dfrac{h(w,\xi) p(w |\alpha)p(\alpha)}{Q(w,\alpha)}dw d\alpha
\end{equation}

After we substitute $Q(w, \alpha)$ with $Q(\alpha)Q(w)$ and we calculate the expectations of 
$alpha$ and $w$ as in the general MFVB, we obtain this expression of  $ \mathcal{\tilde{L}}(Q, \xi) $: 
\begin{equation}
\mathcal{\tilde{L}}(Q, \xi)=\frac{1}{2}w_N^T V_N^{-1} w_N + \frac{1}{2} \ln |V_N| + \sum_{n} \left( \ln \sigma(\xi_n) - \frac{\xi_n}{2} + \lambda(\xi_n)\xi_n^2\right)-\ln \Gamma(a_0) +a_0 \ln(b_0)-b_0\frac{a_N}{b_N}-a_N \ln(b_N)-\ln \Gamma(a_N) +a_N
\end{equation}
with 
$$a_N=a_0 +\frac{D}{2} $$
$$b_N=b_0+\frac{1}{2}w_N^T w_N +Tr(V_N)$$
$$V_N^{-1}=\frac{a_N}{b_N} \mathcal{I} +2 \sum_{n} \lambda(\xi_n)x_n x_n^T$$
$$w_N=V_N \sum_{n}\frac{y_n}{2}x_n $$

and 
$$Q ^ *(w)=\mathcal{N}(w | w_N, V_N)$$ 
$$Q^ *(\alpha)=Gam(\alpha | a_N, b_N)$$
This bound depends on $\xi$. We maximize this bound with respect to $\xi$ and we find: 
\begin{equation}
(\xi_n^{new})^2=x_n^T (V_N +w_Nw_N ^T) x_N
\end{equation}


We use the EM algorithm to update the equations for $ w_N , V_N, a_N, b_N, \text{and}~ \xi $ in order to maximize the variational bound $\mathcal{\tilde{L}}(Q, \xi)$, until it reaches a plateau. 


